% Options for packages loaded elsewhere
\PassOptionsToPackage{unicode}{hyperref}
\PassOptionsToPackage{hyphens}{url}
\PassOptionsToPackage{dvipsnames,svgnames,x11names}{xcolor}
%
\documentclass[
  letterpaper,
  DIV=11,
  numbers=noendperiod]{scrartcl}

\usepackage{amsmath,amssymb}
\usepackage{iftex}
\ifPDFTeX
  \usepackage[T1]{fontenc}
  \usepackage[utf8]{inputenc}
  \usepackage{textcomp} % provide euro and other symbols
\else % if luatex or xetex
  \usepackage{unicode-math}
  \defaultfontfeatures{Scale=MatchLowercase}
  \defaultfontfeatures[\rmfamily]{Ligatures=TeX,Scale=1}
\fi
\usepackage{lmodern}
\ifPDFTeX\else  
    % xetex/luatex font selection
\fi
% Use upquote if available, for straight quotes in verbatim environments
\IfFileExists{upquote.sty}{\usepackage{upquote}}{}
\IfFileExists{microtype.sty}{% use microtype if available
  \usepackage[]{microtype}
  \UseMicrotypeSet[protrusion]{basicmath} % disable protrusion for tt fonts
}{}
\makeatletter
\@ifundefined{KOMAClassName}{% if non-KOMA class
  \IfFileExists{parskip.sty}{%
    \usepackage{parskip}
  }{% else
    \setlength{\parindent}{0pt}
    \setlength{\parskip}{6pt plus 2pt minus 1pt}}
}{% if KOMA class
  \KOMAoptions{parskip=half}}
\makeatother
\usepackage{xcolor}
\setlength{\emergencystretch}{3em} % prevent overfull lines
\setcounter{secnumdepth}{-\maxdimen} % remove section numbering
% Make \paragraph and \subparagraph free-standing
\ifx\paragraph\undefined\else
  \let\oldparagraph\paragraph
  \renewcommand{\paragraph}[1]{\oldparagraph{#1}\mbox{}}
\fi
\ifx\subparagraph\undefined\else
  \let\oldsubparagraph\subparagraph
  \renewcommand{\subparagraph}[1]{\oldsubparagraph{#1}\mbox{}}
\fi


\providecommand{\tightlist}{%
  \setlength{\itemsep}{0pt}\setlength{\parskip}{0pt}}\usepackage{longtable,booktabs,array}
\usepackage{calc} % for calculating minipage widths
% Correct order of tables after \paragraph or \subparagraph
\usepackage{etoolbox}
\makeatletter
\patchcmd\longtable{\par}{\if@noskipsec\mbox{}\fi\par}{}{}
\makeatother
% Allow footnotes in longtable head/foot
\IfFileExists{footnotehyper.sty}{\usepackage{footnotehyper}}{\usepackage{footnote}}
\makesavenoteenv{longtable}
\usepackage{graphicx}
\makeatletter
\def\maxwidth{\ifdim\Gin@nat@width>\linewidth\linewidth\else\Gin@nat@width\fi}
\def\maxheight{\ifdim\Gin@nat@height>\textheight\textheight\else\Gin@nat@height\fi}
\makeatother
% Scale images if necessary, so that they will not overflow the page
% margins by default, and it is still possible to overwrite the defaults
% using explicit options in \includegraphics[width, height, ...]{}
\setkeys{Gin}{width=\maxwidth,height=\maxheight,keepaspectratio}
% Set default figure placement to htbp
\makeatletter
\def\fps@figure{htbp}
\makeatother
\newlength{\cslhangindent}
\setlength{\cslhangindent}{1.5em}
\newlength{\csllabelwidth}
\setlength{\csllabelwidth}{3em}
\newlength{\cslentryspacingunit} % times entry-spacing
\setlength{\cslentryspacingunit}{\parskip}
\newenvironment{CSLReferences}[2] % #1 hanging-ident, #2 entry spacing
 {% don't indent paragraphs
  \setlength{\parindent}{0pt}
  % turn on hanging indent if param 1 is 1
  \ifodd #1
  \let\oldpar\par
  \def\par{\hangindent=\cslhangindent\oldpar}
  \fi
  % set entry spacing
  \setlength{\parskip}{#2\cslentryspacingunit}
 }%
 {}
\usepackage{calc}
\newcommand{\CSLBlock}[1]{#1\hfill\break}
\newcommand{\CSLLeftMargin}[1]{\parbox[t]{\csllabelwidth}{#1}}
\newcommand{\CSLRightInline}[1]{\parbox[t]{\linewidth - \csllabelwidth}{#1}\break}
\newcommand{\CSLIndent}[1]{\hspace{\cslhangindent}#1}

\KOMAoption{captions}{tableheading}
\makeatletter
\@ifpackageloaded{tcolorbox}{}{\usepackage[skins,breakable]{tcolorbox}}
\@ifpackageloaded{fontawesome5}{}{\usepackage{fontawesome5}}
\definecolor{quarto-callout-color}{HTML}{909090}
\definecolor{quarto-callout-note-color}{HTML}{0758E5}
\definecolor{quarto-callout-important-color}{HTML}{CC1914}
\definecolor{quarto-callout-warning-color}{HTML}{EB9113}
\definecolor{quarto-callout-tip-color}{HTML}{00A047}
\definecolor{quarto-callout-caution-color}{HTML}{FC5300}
\definecolor{quarto-callout-color-frame}{HTML}{acacac}
\definecolor{quarto-callout-note-color-frame}{HTML}{4582ec}
\definecolor{quarto-callout-important-color-frame}{HTML}{d9534f}
\definecolor{quarto-callout-warning-color-frame}{HTML}{f0ad4e}
\definecolor{quarto-callout-tip-color-frame}{HTML}{02b875}
\definecolor{quarto-callout-caution-color-frame}{HTML}{fd7e14}
\makeatother
\makeatletter
\makeatother
\makeatletter
\makeatother
\makeatletter
\@ifpackageloaded{caption}{}{\usepackage{caption}}
\AtBeginDocument{%
\ifdefined\contentsname
  \renewcommand*\contentsname{Table of contents}
\else
  \newcommand\contentsname{Table of contents}
\fi
\ifdefined\listfigurename
  \renewcommand*\listfigurename{List of Figures}
\else
  \newcommand\listfigurename{List of Figures}
\fi
\ifdefined\listtablename
  \renewcommand*\listtablename{List of Tables}
\else
  \newcommand\listtablename{List of Tables}
\fi
\ifdefined\figurename
  \renewcommand*\figurename{Figure}
\else
  \newcommand\figurename{Figure}
\fi
\ifdefined\tablename
  \renewcommand*\tablename{Table}
\else
  \newcommand\tablename{Table}
\fi
}
\@ifpackageloaded{float}{}{\usepackage{float}}
\floatstyle{ruled}
\@ifundefined{c@chapter}{\newfloat{codelisting}{h}{lop}}{\newfloat{codelisting}{h}{lop}[chapter]}
\floatname{codelisting}{Listing}
\newcommand*\listoflistings{\listof{codelisting}{List of Listings}}
\makeatother
\makeatletter
\@ifpackageloaded{caption}{}{\usepackage{caption}}
\@ifpackageloaded{subcaption}{}{\usepackage{subcaption}}
\makeatother
\makeatletter
\@ifpackageloaded{tcolorbox}{}{\usepackage[skins,breakable]{tcolorbox}}
\makeatother
\makeatletter
\@ifundefined{shadecolor}{\definecolor{shadecolor}{rgb}{.97, .97, .97}}
\makeatother
\makeatletter
\makeatother
\makeatletter
\makeatother
\ifLuaTeX
  \usepackage{selnolig}  % disable illegal ligatures
\fi
\IfFileExists{bookmark.sty}{\usepackage{bookmark}}{\usepackage{hyperref}}
\IfFileExists{xurl.sty}{\usepackage{xurl}}{} % add URL line breaks if available
\urlstyle{same} % disable monospaced font for URLs
\hypersetup{
  pdftitle={Body fluid identification},
  pdfauthor={Courtney Lynch},
  colorlinks=true,
  linkcolor={blue},
  filecolor={Maroon},
  citecolor={Blue},
  urlcolor={Blue},
  pdfcreator={LaTeX via pandoc}}

\title{Body fluid identification}
\author{Courtney Lynch}
\date{2023-06-27}

\begin{document}
\maketitle
\ifdefined\Shaded\renewenvironment{Shaded}{\begin{tcolorbox}[sharp corners, interior hidden, frame hidden, breakable, boxrule=0pt, enhanced, borderline west={3pt}{0pt}{shadecolor}]}{\end{tcolorbox}}\fi

\hypertarget{introduction}{%
\section{Introduction}\label{introduction}}

\hypertarget{hello}{%
\subsection{hello}\label{hello}}

\begin{itemize}
\tightlist
\item
  hi
\item
  hi
\end{itemize}

\begin{tcolorbox}[enhanced jigsaw, colback=white, arc=.35mm, colframe=quarto-callout-tip-color-frame, bottomrule=.15mm, coltitle=black, rightrule=.15mm, breakable, colbacktitle=quarto-callout-tip-color!10!white, title=\textcolor{quarto-callout-tip-color}{\faLightbulb}\hspace{0.5em}{Markdown exercise}, opacitybacktitle=0.6, bottomtitle=1mm, toptitle=1mm, leftrule=.75mm, left=2mm, opacityback=0, toprule=.15mm, titlerule=0mm]

\begin{itemize}
\item
  Check out
  \href{https://quarto.org/docs/authoring/markdown-basics.html}{Quarto's
  markdown guide}
\item
  and the
  \href{https://biostats-r.github.io/biostats/quarto/04-figures-tables.html\#equations}{biostats
  guide to writing (equations section)}
\end{itemize}

In the template Introduction section take 3-4 minutes to write:

\begin{itemize}
\item
  a sub-heading under the introduction
\item
  a list
\item
  and an equation (inline or as a block) \(y=mx+c\)
\item
  cite one of the articles in in existing \texttt{.bib} file using the
  \texttt{@} Chamberlin (1897)
\item
  remove this callout and render the document and hit the
  :heavy\_check\_mark: in zoom
\end{itemize}

\end{tcolorbox}

\hypertarget{libraries}{%
\section*{Libraries}\label{libraries}}

Generally it is good practice to include a list of packages you use up
front. But you may not need to show them in the output or list them in
the table of contents. The \texttt{\{.unnumbered\ .unlisted\}} commands
following the heading remove this section from the table of contents but
it will remain in the text. Quarto accepts multiple coding languages,
the following example uses R. I have set up working R code chunks so
that for this tutorial you do not need to be familiar with R. Here, we
will experiment with a few code
\href{https://quarto.org/docs/computations/execution-options.html}{execution
options}.

If you want to use Python code check out the documentation
\href{https://quarto.org/docs/computations/python.html\#overview}{here}.

\begin{tcolorbox}[enhanced jigsaw, colback=white, arc=.35mm, colframe=quarto-callout-tip-color-frame, bottomrule=.15mm, coltitle=black, rightrule=.15mm, breakable, colbacktitle=quarto-callout-tip-color!10!white, title=\textcolor{quarto-callout-tip-color}{\faLightbulb}\hspace{0.5em}{Code block exercise}, opacitybacktitle=0.6, bottomtitle=1mm, toptitle=1mm, leftrule=.75mm, left=2mm, opacityback=0, toprule=.15mm, titlerule=0mm]

\begin{itemize}
\item
  Check out the
  \href{https://quarto.org/docs/computations/execution-options.html}{Block
  options} and edit (and add to) the existing code block options below
  to:
\item
  exclude the results, messages and warnings
\item
  exclude the code from the HTML output
\item
  render the document and hit the :heavy\_check\_mark: in zoom
\end{itemize}

\end{tcolorbox}

\hypertarget{tabsets}{%
\section{Tabsets}\label{tabsets}}

\section{Tab 1}

blah blah

\section{Tab 2}

blah

\begin{tcolorbox}[enhanced jigsaw, colback=white, arc=.35mm, colframe=quarto-callout-tip-color-frame, bottomrule=.15mm, coltitle=black, rightrule=.15mm, breakable, colbacktitle=quarto-callout-tip-color!10!white, title=\textcolor{quarto-callout-tip-color}{\faLightbulb}\hspace{0.5em}{Tabset exercise}, opacitybacktitle=0.6, bottomtitle=1mm, toptitle=1mm, leftrule=.75mm, left=2mm, opacityback=0, toprule=.15mm, titlerule=0mm]

\begin{itemize}
\item
  Check out the
  \href{https://quarto.org/docs/interactive/layout.html\#tabset-panel}{tabset
  panel documentation}
\item
  Create a tabset with three tabs in the template under the Tabsets
  heading

  \begin{itemize}
  \tightlist
  \item
    Does not matter what they contain but feel free to blurb something
    in there
  \end{itemize}
\item
  render the document and hit the :heavy\_check\_mark: in zoom
\end{itemize}

\end{tcolorbox}

\hypertarget{images}{%
\section{Images}\label{images}}

\begin{tcolorbox}[enhanced jigsaw, colback=white, arc=.35mm, colframe=quarto-callout-tip-color-frame, bottomrule=.15mm, coltitle=black, rightrule=.15mm, breakable, colbacktitle=quarto-callout-tip-color!10!white, title=\textcolor{quarto-callout-tip-color}{\faLightbulb}\hspace{0.5em}{Images exercise}, opacitybacktitle=0.6, bottomtitle=1mm, toptitle=1mm, leftrule=.75mm, left=2mm, opacityback=0, toprule=.15mm, titlerule=0mm]

\begin{itemize}
\item
  There is a (royalty free) XKCD comic inside the images directory in
  the repo. using the syntax described above, insert the image into one
  of the tabsets you just created.
\item
  render the document and hit the :heavy\_check\_mark: in zoom
\end{itemize}

\end{tcolorbox}

\hypertarget{letg-go-git-it}{%
\subsection{let'g go Git it}\label{letg-go-git-it}}

You're now ready to host your first live link!

\begin{tcolorbox}[enhanced jigsaw, colback=white, arc=.35mm, colframe=quarto-callout-tip-color-frame, bottomrule=.15mm, coltitle=black, rightrule=.15mm, breakable, colbacktitle=quarto-callout-tip-color!10!white, title=\textcolor{quarto-callout-tip-color}{\faLightbulb}\hspace{0.5em}{GitHub}, opacitybacktitle=0.6, bottomtitle=1mm, toptitle=1mm, leftrule=.75mm, left=2mm, opacityback=0, toprule=.15mm, titlerule=0mm]

\begin{itemize}
\item
  Render your project so that most recent changes are exported
\item
  In the source control on the left commit and push your changes

  \begin{itemize}
  \tightlist
  \item
    You can do this however you prefer, commandline, GUI, source
    control\ldots{}
  \end{itemize}
\item
  Head over to GitHub in your browser and go:

  \begin{itemize}
  \tightlist
  \item
    Settings -\textgreater{} GitHub pages -\textgreater{} enable github
    pages
  \end{itemize}
\item
  By default, your link will be hosted at:
  https://githubusername.github.io/reponame/pathtodocument.html

  \begin{itemize}
  \tightlist
  \item
    The repo can remain private but anyone who has the link can view it.
  \end{itemize}
\item
  Find your hosted template link and share with the world!
\end{itemize}

\end{tcolorbox}

\hypertarget{references}{%
\section{References}\label{references}}

References are generated by default so include a final empty heading
(delete this text) called References or Bibliography, or whatever is
appropriate.

\hypertarget{refs}{}
\begin{CSLReferences}{1}{0}
\leavevmode\vadjust pre{\hypertarget{ref-chamberlin1897}{}}%
Chamberlin, T. C. 1897. {``The {Method} of {Multiple Working
Hypotheses}.''} \emph{The Journal of Geology} 5: 837--48.

\end{CSLReferences}



\end{document}
